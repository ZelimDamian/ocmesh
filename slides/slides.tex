\documentclass[utf8x]{beamer}

\usepackage{amsmath,amssymb}
\usepackage{hyperref}
\usepackage{listings}
\usepackage{verbatim}
\usepackage{mathtools}
\usepackage[T1]{fontenc}
\usepackage[scaled]{beramono}

\usetheme{Pittsburgh}

\mathtoolsset{showonlyrefs}

\lstset{language=[ISO]C++, basicstyle=\small\ttfamily}

\title{OcMesh}
\subtitle{Towards a hexahedral mesh generator}
\author{Nicola Gigante}
\date{January 22, 2015}

\begin{document}
\begin{frame}
\maketitle
\end{frame}

\section{Introduction}
\begin{frame}[fragile]{What are we talking about?}
\verb|OcMesh|: a C++ tool and library for the generation of hexahedral meshes
from CSG objects.
\vfill
What will we talk about today:
\begin{itemize}
\item What does OcMesh do?
\item What are CSG objects
\item OcMesh uses octrees to generate the mesh:
      \begin{itemize}
      \item What are octrees
      \item How to represent them (efficiently)
      \item How to build them
      \item How to use them
      \end{itemize}
\end{itemize}
\end{frame}

\begin{frame}{The goal}
\begin{itemize}
\item Finite elements methods usually operate on tetrahedral meshes 
      (triangle faces).
\item Discretization of Maxwell constitutive relations on hexahedral meshes 
      are introduced in \cite{Specogna2010}.
\item Hexahedral meshes are missing.
\end{itemize}
\end{frame}

\begin{frame}{Constructive Solid Geometry}
The input geometry is specified as Constructive Solid Geometry:
\begin{quote}
A technique of solid modeling that allows the modeler to create a complex 
surface or object by using Boolean operators to combine smaller objects.
\end{quote}
\end{frame}

\begin{frame}{Constructive Solid Geometry}
We model the world starting from a couple of primitives:
\begin{itemize}
\item Cubes
\item Spheres
\item Possibly others: torus?
\end{itemize}
Combined with a few operations:
\begin{itemize}
\item Union
\item Intersection
\item Difference
\item Affine transforms
\end{itemize}
\end{frame}

\section{Bibliography}
\subsection{}
\begin{frame}[allowframebreaks]
\frametitle{Bibliography}
\begin{center}
\bibliographystyle{apalike}
\bibliography{biblio}
\end{center}
\end{frame}

\end{document}
